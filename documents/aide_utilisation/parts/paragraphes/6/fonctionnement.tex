\subsection{Droits sur un projet}

Par rapport à un projet, nous avons différents participants. On peut regrouper ces participants en 3 niveaux.
\begin{itemize}
    \item \textbf{Le créateur du projet}, ou super-administrateur, est l'utilisateur qui a décidé de créer le projet. Il possède tous les droits sur ce projet et est le seul capable de supprimer le projet.
    \item \textbf{Les administrateurs} sont des utilisateurs qui possèdent différents droits sur ce projet. Ils peuvent accepter ou refuser les demandes pour rejoindre le projet. Ils peuvent agir sur les caractéristiques d'un membre standard du projet. Ils n'ont pas les droits pour agir directement sur les autres administrateurs du projet.
    \item \textbf{Membre standard} regroupe les utilisateurs qui participent au projet et qui n'ont pas de droit particulier.
\end{itemize}

\framebox{
    \begin{minipage}{\dimexpr\textwidth-2\fboxsep-2\fboxrule\relax}
        \textbf{Remarque :}\\
        Les niveaux de droits sur un projet sont indépendants des niveaux de droits d'un auteur SPIP. Les administrateurs du site SPIP ont des droits de \textit{super-administrateur} sur les projets avec l'option 7. Les rédacteurs et visiteurs ont des droits standards, c'est-à-dire ceux définis directement dans le projet. 
    \end{minipage}
}

\subsection{Conditions et récompenses}

Une des possibilités d'utilisation de ce plugin est de voir les projets comme des formations. En voyant cela ainsi, il peut être logique d'imaginer des conditions d'entrée et des récompenses par rapport à la participation à un projet. Cette utilisation est possible via un système de mots-clés présenté dans l'option 4.

\subsection{Récursivité}

Une des possibilités est de permettre la participation à des projets, mais dans un ordre précis. Cette manière de voir les choses se rapproche de l'utilisation de l'option 6 (un utilisateur ne peut pas rejoindre un projet tant qu'il n'a pas validé le projet racine).\\
\newline
Dans le cas où un projet demande déjà des critères d'entrée, la récurrence des projets est une condition supplémentaire à valider pour participer au projet.
