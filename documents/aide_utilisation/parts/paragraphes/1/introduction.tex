Nous allons ici présenter le plugin Spip nommé \textbf{OSI Projets}. Ce plugin a pour objectif de pouvoir mettre en place un système de gestion de projets au sein d'un site Spip. La version présentée ici est la version \texttt{1.1.28} qui est encore une version de développement. C'est à-dire que le projet peut encore posséder certains bugs qu'il ne faut pas hésiter à déclarer, voire à directement corriger via le GitHub. À savoir que le plugin est accessible sur la page GitHub suivante :
\begin{center}
    \href{https://github.com/TuilierGit/osi_projets}{https://github.com/TuilierGit/osi\_projets}
\end{center}


\subsection{Histoire du plugin}

Ce plugin a été réalisé à la demande de l'ONG "OBJECTIF SCIENCES International" (OSI). Il a été réalisé dans sa première version durant un stage en 2024 par Thomas Tuilier.

\subsection{Objectifs du plugin}
Les objectifs de ce plugin sont les suivants : 
\begin{itemize}
    \item Les projets doivent pouvoir regrouper différents participants, chacun ayant plus ou moins de droits sur les projets.
    \item Chaque projet doit pouvoir ajouter des critères d’entrée et définir un certain agencement entre eux, c’est-à-dire un ordre de participation aux projets.
    \item Les projets peuvent être considérés comme des formations, ils doivent donc permettre de définir des récompenses pout les participants. 
    \item Il doit être possible d'ajouter des rôles aux membres d'un projet.
    \item Un projet doit pouvoir être visible sur le front du site.
\end{itemize}

